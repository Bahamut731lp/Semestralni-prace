\chapter{Slovník pojmů}
\label{app:slovnik}
\noindent\begin{longtable}[l]{@{}lp{12cm}@{}}
	\textbf{Programování} & Proces vytvoření programu, který řeší nějaký problém. \\
	\textbf{Algoritmus} & Konečný sled instrukcí, který vede k řešení problému, který má algoritmus řešit. \\
	\textbf{Algoritmizace} & Proces vytváření algoritmu. \\
	\textbf{Asymptotická složitost} & Funkce, která nám říká, jak rychle roste nějaká veličina společně s velikostí vstupních dat. Platí, že čím pomaleji roste, tím je algoritmus efektivnější. \\
	\textbf{Datový typ} & Určení, jakým způsobem budou data uložena v paměti, a co s takovými daty můžeme provádět. \\
	\textbf{Přetypování} & Proces převodu mezi datovými typy, pokud je takový převod možný. \\
	\textbf{Výraz} & Sekvence konstant a proměnných, které jsou spojeny operátory a které jsou vyhodnocovány. \\
	\textbf{Příkaz} & Volání procedury nebo funkce, které jsou vykonávány. \\
	\textbf{Operátor} & Určuje vztah mezi věcí nalevo od něj a napravo od něj.\\
	\textbf{Procedura} & Spustitelný blok kódu, který obsahuje sled instrukcí.\\
	\textbf{Funkce} & Procedura, která na konci (nebo někdy během svého běhu) vrátí nějaký výsledek.\\
	\textbf{Podmíněné větvení} & Struktura, která podle pravdivosti výrazu rozhodne, který blok kódu má spustit.\\
	\textbf{Cyklus} & Struktura, která slouží k opakování bloků kódu.\\
	\textbf{Tělo cyklu} & Blok kódu, který bude smyčkou opakovaně vykonáván.\\
	\textbf{Iterace} & Jedno vykonání kódu v těle cyklu.\\
	\textbf{Pole} & Datová struktura, která má fixní počet prvků. Tyto prvky jsou řazeny za sebou a lze k nim přistupovat pomocí indexu.\\
	\textbf{Index} & Pořadové číslo prvku v poli.\\
	\textbf{Asociativní pole} & Datová struktura, která je složená z dvojic, klíč a hodnota. K hodnotě se přistupuje pomocí klíče.\\
	\textbf{Výjimka} & Událost, která nastane v případě, kdy program nemůže nějakou operaci/instrukci provést. \\
	\textbf{Chyba} & Událost, která nastane při selhání prostředků počítače za běhu programu. \\
	\textbf{Objekt} & Základní jednotka objektově orientovaného programování, která si uchovává \textbf{atributy} (data) a \textbf{metody} (operace, které dokáže provést). \\
	\textbf{Atribut} & Informace (ať už proměnná či konstantí), kterou si objekt uchovává v paměti. \\
	\textbf{Metoda} & Pojmenovaný blok kódu v danému objektu, který může objekt volat. \\
	\textbf{Třída} & Předepis objektu, jeho atributů a metod, společně s jejich datovými typy a modifikátory přístupu. \\
	\textbf{Konstruktor} & Speciální metoda třídy, která se volá při vytváření objektu. \\
	\textbf{Instance třídy} & Objekt, který byl vytvořen pomocí konstruktoru nějaké třídy. \\
	\textbf{Modifikátory přístupu} & Sada klíčových slov, které říkají, jakým způsobem lze ke struktuře přistupovat z vnějšku. \\
	\textbf{Abstrakce} & Proces, který složité úkony schová za jednoduché rozhraní. \\
	\textbf{Zapouzdření} & Koncept, kdy pomocí modifikátorů přístupu kontrolujeme, jak a co bude přístupné z vnějšího programu. \\
	\textbf{Skládání objektů} & Proces, ve kterém v metodě jednoho objektu používáme prostředky cizího objektu. \\
	\textbf{Statická deklarace} & Vytvoření (deklarace) struktury, jejíž hodnota není závislá na vnitřním stavu instance objektu. \\ 
	\textbf{Dědičnost} & Koncept dovolující vytvářet nové objekty a struktury na základě již dříve definovaných objektů/struktur. \\
	\textbf{Polymorfismus} & Vlastnost objektů, která říká, že nezáleží na typu dat, jelikož je objekt dokáže obsloužit.
\end{longtable}
